\documentclass[10pt,landscape]{article}
\usepackage{amssymb,amsmath,amsthm,amsfonts}
\usepackage{multicol,multirow}
\usepackage{calc}
\usepackage{color}
\usepackage{listings}
\usepackage{caption}
\usepackage{ifthen}
\usepackage[utf8]{inputenc}
\usepackage[landscape]{geometry}
\usepackage[colorlinks=true,citecolor=blue,linkcolor=blue]{hyperref}


%\newcounter{nalg}[chapter] % defines algorithm counter for chapter-level
%\renewcommand{\thenalg}{\thechapter .\arabic{nalg}} %defines appearance of the algorithm counter
\DeclareCaptionLabelFormat{algocaption}{Algorithm \thenalg} % defines a new caption label as Algorithm x.y

\lstnewenvironment{algorithm}[1][] %defines the algorithm listing environment
{   
    %\refstepcounter{nalg} %increments algorithm number
    %\captionsetup{labelformat=algocaption,labelsep=colon} %defines the caption setup for: it ises label format as the declared caption label above and makes label and caption text to be separated by a ':'
    \lstset{ %this is the stype
        mathescape=true,
        %frame=tB,
        %numbers=left, 
        numberstyle=\tiny,
        basicstyle=\scriptsize, 
        keywordstyle=\color{black}\bfseries\em,
        keywords={,vhod, izhod, vrni, dokler, izvajaj, konec, zanke} %add the keywords you want, or load a language as Rubens explains in his comment above.
        %numbers=left,
        %xleftmargin=.04\textwidth,
        %#1 % this is to add specific settings to an usage of this environment (for instnce, the caption and referable label)
    }
}
{}

\ifthenelse{\lengthtest { \paperwidth = 11in}}
    { \geometry{top=.5in,left=.5in,right=.5in,bottom=.5in} }
	{\ifthenelse{ \lengthtest{ \paperwidth = 297mm}}
		{\geometry{top=1cm,left=1cm,right=1cm,bottom=1cm} }
		{\geometry{top=1cm,left=1cm,right=1cm,bottom=1cm} }
	}
\pagestyle{empty}
\makeatletter
\renewcommand{\section}{\@startsection{section}{1}{0mm}%
                                {-1ex plus -.5ex minus -.2ex}%
                                {0.5ex plus .2ex}%x
                                {\normalfont\large\bfseries}}
\renewcommand{\subsection}{\@startsection{subsection}{2}{0mm}%
                                {-1explus -.5ex minus -.2ex}%
                                {0.5ex plus .2ex}%
                                {\normalfont\normalsize\bfseries}}
\renewcommand{\subsubsection}{\@startsection{subsubsection}{3}{0mm}%
                                {-1ex plus -.5ex minus -.2ex}%
                                {1ex plus .2ex}%
                                {\normalfont\small\bfseries}}
\makeatother
\setcounter{secnumdepth}{0}
\setlength{\parindent}{0pt}
\setlength{\parskip}{0pt plus 0.5ex}



% -----------------------------------------------------------------------

\title{Diskretne strukture}

\begin{document}

\raggedright
\footnotesize

\begin{multicols}{3}
\setlength{\premulticols}{1pt}
\setlength{\postmulticols}{1pt}
\setlength{\multicolsep}{1pt}
\setlength{\columnsep}{2pt}

\section{Relacije}
Naj bo $A$ neka množica. Podmnožici $R \subseteq A \times A$ rečemo \emph{relacija} na množici $A$.
Če je $(x, y) \in R$, pišemo $xRy$.

\[R(x) = \{y \in A : xRy\}\]
\[R^{-1}(y) = \{x \in A : yRx\}\]

\emph{Definicijsko območje:}
\[D_R = \{x \in A: R(x) \neq 0\}\]
\emph{Zaloga vrednosti:}
\[Z_R = \{y \in A: R^{-1}(y) \neq 0\}\]

\emph{Graf relacije} $R \subseteq A^2$ je slika na kateri vsakemu elementu iz $A$ pripada svoja točka za vask par $(x, y) \in R$ pa naredimo puščico iz $x \to y$.

\subsection{Operacije na relacijah}
\emph{Komplement}
\[\overline{R} = A^2 - R \]
\emph{Inverz}
\[ xR^{-1}y \Leftrightarrow yRy\]
\emph{Kompozitum}
 \[x(R\circ T)y \Leftrightarrow \exists z \in A : xTz \wedge zRy \]

\subsection{Lastnosti relacij}

$R$ je \emph{refleksinva} $\Leftrightarrow$ $\forall x \in A : xRx$

$R$ je \emph{irefleksinva} $\Leftrightarrow$ $\forall x \in A : \neg (xRx) $

$R$ je \emph{simetrična} $\Leftrightarrow$ $\forall x, y \in A : xRy \Rightarrow yRx$

$R$ je \emph{asimetrična} $\Leftrightarrow$ $\forall x, y \in A : xRy \Rightarrow \neg (yRx)$

$R$ je \emph{antisimetrična} $\Leftrightarrow$ $\forall x, y \in A : xRy \Rightarrow \neg (yRx) \vee x=y$

$R$ je \emph{tranzitivna} $\Leftrightarrow$ $\forall x, y, z \in A : xRy \wedge yRz \Rightarrow xRz$

$R$ je \emph{sovisna} $\Leftrightarrow$ $\forall x, y \in A : xRy \vee yRx \vee x=y$

$R$ je \emph{strogosovisna} $\Leftrightarrow$ $\forall x, y \in A : xRy \vee yRx$

$R$ je \emph{enolična} $\Leftrightarrow$ $\forall x, y, z \in A : xRy \wedge xRz \Rightarrow y=z$

\subsection{Ekvivalenčne relacije}
Relacija je ekvivalenčna, če je \emph{refleksivna}, \emph{simetrična} in \emph{tranzitivna}.
\subsubsection{Ekvivalenčni razred}
Naj bo $R$ ekvivalenčna relacija. $R(a)$ je ekvivalenčni razred elementa $a$.
\[a \in A: R(a) = \{b \in A: aRb\}\]
Množica ekvivalenčnih razredov glede na $R$:
\[A_{/R} = \{R(a): a \in A\}\]

\subsection{Relacije urejenosti}
Vsaki \emph{tranzitivni} relaciji rečemo relacija urejenosti. Naj bo $R$ tranzitivno:
\begin{itemize}
    \item $R$ je \textbf{delna urejenost}, če je \emph{refleksivna} in \emph{antisimetrična}
    \item $R$ je \textbf{linearna urejenost}, če je \emph{antisimetrična} in \emph{strogo sovisna} (in zato tudi refleksivna)
    \item $R$ je \textbf{stroga delna urejenost}, če je \emph{asimetrilna} (in zato irefleksivna)
    \item $R$ je \textbf{stroga linearna urejenost}, če je \emph{asimetrilna} in \emph{sovisna}
    \item $R$ je \textbf{dobra urejenost}, če je \emph{linearna urejenost} in ima vasaka podmnožica svoj \emph{minimum}
\end{itemize}

\bigbreak
Naj bo $R$ relacija urejenosti na $A$ in $X \subseteq A$:

\begin{itemize}
    \item Element $a \in A$ je \textbf{zgornja meja} za $X$, če velja $\forall x \in X: xRa$.
    \item Element $a \in A$ je \textbf{spodnja meja} za $X$, če velja $\forall x \in X: aRx$.
    \item Zgornja meja $a \in X$ je \textbf{natančna zgornja meja}/supremum, če za vsako zgornjo mejo $b$ množice $X$ velja $a=b \vee aRb$.
    \item Spodnja meja $a \in X$ je \textbf{natančna spodnja meja}/infimum, če za vsako spodnjo mejo $b$ množice $X$ velja $a=b \vee bRa$.
    \item \textbf{maksimum} je taka \emph{natančna zgornja meja}, ki je vsebovana v množici $X$.
    \item \textbf{minimum} je taka \emph{natančna spodnja meja}, ki je vsebovana v množici $X$.
\end{itemize}

\section{Funkcije}
\emph{Enolični relaciji} $R$ na $A$ rečemo tudi funkcija.
\[R: D_R \rightarrow A\]

Relacija $R \subseteq A^2$ je:
\begin{itemize}
    \item \textbf{injektivna} $\Leftrightarrow$ $\forall x, y, z \in A : xRy \wedge zRy \Rightarrow x=z$
    \item \textbf{surjektivna} $\Leftrightarrow$ $Z_R = A$
    \item \textbf{bijektivna} $\Leftrightarrow$ injektivna in surjektivna
\end{itemize}

Funkcija $R$ ima inverz $\Leftrightarrow$ ko je $R$ \emph{injektivna}

\subsection{Moč množic}
Množici $A$, $B$ sta eneko močni (ekvipotentni, imata isto kardinalnost), če obstaja bijektivna preslikava iz $A$ v $B$. Pišemo $|A|=|B|$.

Množica $A$ je \emph{neskončna} $\Leftrightarrow$ $\exists B \subset A : |A|=|B|$

\[|A| \leq |B| \Leftrightarrow \exists f : A \rightarrow B\text{, ki je injektivna}\]
\[|A| \leq |B| \Leftrightarrow \exists f : B \rightarrow A\text{, ki je surjektivna}\]


\section{Teorija števil}
\[a|b \Leftrightarrow \exists k : b = ka\]
Deleitelji števila $a$:
\[D(a) = \{m \in \mathbb{Z} : m|a\}\]
\[D^+(a) = \{m \in \mathbb{N} : m|a\}\]
Večkratniki števila $a$:
\[V(a) = \{b \in \mathbb{Z} : a|b\} = \{ka: k \in \mathbb{Z}\}\]
\emph{Največji skupni delitelj}
\[\textrm{gcd}(a, b) = \textrm{max}(D^+(a) \cap D^+(b))\]
\emph{Najmanjši skupni večkratniki}
\[\textrm{lcm}(a, b) = \textrm{min}(V^+(a) \cup V^+(b))\]

Če za $a, b \in \mathbb{Z} - \{0\}$ velja $\textrm{gcd}(a, b) = 1$, sta si $a$ in $b$ \textbf{tuji števili}.

\[ \textrm{gcd}(a, b) \cdot \textrm{lcm}(a, b) = a \cdot b \]

\[a, b, c \in \mathbb{Z} - \{0\} : \textrm{gcd}(a, b) = 1 \wedge a | bc \Rightarrow a | c\]
\[a, b, c \in \mathbb{Z} - \{0\} : c | a \wedge c | b \Rightarrow \textrm{gcd} \left( \frac{a}{c}, \frac{b}{c} \right) = \frac{\textrm{gcd}(a,b)}{c}\]
\[\textrm{gcd} \left( \frac{a}{\textrm{gcd}(a, b)}, \frac{b}{\textrm{gcd}(a, b)} \right) = 1\]

\subsubsection{Praštevila}
Če sta edina pozitivna delitelja naravnega števila $n \geq 2$ $1$ in $n$, je $n$ \textbf{praštevilo}. Množica preštevil:
\[\mathbb{P} = \{2,3,5,7,11,...\}\]

\subsubsection{Razcep na prafaktorje}
Vsak $n \geq 2$ lahko zapišemo kot produkt praštevil $p_1,..., p_m$:
\[n = p_1^{\alpha_1} \cdot p_2^{\alpha_2} \cdot ... \cdot p_m^{\alpha_m}\]

\subsection{Linearne diofantske enačbe}
Diofantska enačba $ax + by = c$ ima rešitev $\Leftrightarrow$ $gcd(a, b) | c$. 

Če ima eno rešitev $(x_0, y_0) \in \mathbb{Z}^2$ ima neskončno množico rešitev:
\[\{(x_k, y_k) : k \in \mathbb{Z}\}\]
\[x_k = x_0 - k\frac{b}{\textrm{gcd(a, b)}}\]
\[y_k = y_0 + k\frac{a}{\textrm{gcd(a, b)}}\]

\subsubsection{Razširjen evklidov algoritem}

\begin{algorithm}
vhod: $(a, b)$
($r_0$, $x_0$, $y_0$) = ($a$, 1, 0)
($r_1$, $x_1$, $y_1$) = ($b$, 0, 1)
$i$ = 1

dokler $r_i$ $\neq$ 0:
    $i$ = $i$+1
    $k_i$ = $r_{i-2} // r_{i-1}$
    $(r_i, x_i, y_i)$ = $(r_{i-2}, x_{i-2}, y_{i-2}) - k_i(r_{i-1}, x_{i-1}, y_{i-1})$
konec zanke
vrni: $(r_{i-1}, x_{i-1}, y_{i-1})$
\end{algorithm}

Naj bosta $a, b \in \mathbb{Z}$. Tedaj trojica $(d, x, y)$, ki jo vrne razširjen evklidov algoritem z vhodnim podatkomk $(a, b)$, zadošča:
\[ax + by = d \text{ in } d = \textrm{gcd}(a, b)\] 

\section{Modularna aritmetika}
\emph{Kongurenca}
\[a \equiv_m b \Leftrightarrow m | (b-a)\]

\[a \equiv_m b \Leftrightarrow a \textrm{ mod } m = b \textrm{ mod } m \]

\[r = x \textrm{ mod } m \Leftrightarrow r \equiv_m x \text{ in } r \in \{0,1,...,m-1\} \]

Če je $x_1 \equiv_m y_1$ in $x_2 \equiv_m y_2$:
\[x_1 + x_2 \equiv_m y_1 + y_2\]
\[x_1 x_2 \equiv_m y_1 y_2\]
\[x_1^r \equiv_m y_1^r\]

Če je $ax \equiv_m ay$:
\[x \equiv y \left(\textrm{mod} \frac{m}{\textrm{gcd}(a, m)}\right)\]
\subsection{Kolobar ostankov}
\[\mathbb{Z}_m = \{0,1,...,m-1\}\]
\[a, b \in \mathbb{Z}_m\]

$a \bigoplus b = (a + b) \textrm{ mod } m \text{ $\sim$ } a + b$

$a \bigodot b = (a b) \textrm{ mod } m \text{ $\sim$ } ab$ 

\bigbreak
$(\mathbb{Z}_m, +, \cdot)$ je kolobar ostankov po mod $m$
\begin{itemize}
    \item Operaciji $+$ in $\cdot$ sta asociativni, distributivni in komutativni
    \item $0$ je enota za $+$ in $1$ je enota za $\cdot$
    \item vsak $a \in \mathbb{Z}_m$ ima nasprotni element ($-a$)
\end{itemize}
\[ -a = 
\begin{cases}
m-a; a \neq 0 \\
0; a = 0
\end{cases}
\]

Naj bo $a \in \mathbb{Z}_m$. Če obstaja $b \in \mathbb{Z}_m$, za katerega je
$ab = 1$ v $\mathbb{Z}_m$
potem je $a$ \emph{obrnljiv} in $b$ njegov \emph{inverz} (v $\mathbb{Z}_m$).

Množico vseh obrnlivih elementov v $\mathbb{Z}_m$ označimo $\mathbb{Z}_m^*$.
\[a \in \mathbb{Z}_m^* \Leftrightarrow a \text{ je tuj } m\]

Inverz od $a$ je tisti $x \in \mathbb{Z}_m$, ki (skupaj z nekim $y$) reši diofantsko enačbo $ax + (-m)y = 1$ 

Vsak element $\mathbb{Z}_m^*$ ima natanko en inverz. Označimo ga z $a^{-1}$.

\subsection{Euljerjeva funkcija}
Euljerjeva funkcija nam pove koliko je obrnlivih elementov v $\mathbb{Z}_m$.
\[\varphi(m) = 
\begin{cases}
|\{a \in \mathbb{Z}_m - \{0\} : \textrm{gcd}(a,m)=1\}|; m \geq 2 \\
1; m = 1
\end{cases}
\]

\[\varphi(p^k) = (p-1)p^{k-1} = p^k\left(1-\frac{1}{p}\right); p \in \mathbb{P}\]

Za $n \in \mathbb{N}$ s paraštevilskim razcepom $ n = p_1^{\alpha_1} \cdot ... \cdot p_m^{\alpha_m}$ velja:
\[\varphi(n) = \varphi(p_1^{\alpha_1}) \cdot ... \cdot \varphi(p_m^{\alpha_m}) = n \prod_{ p_k \in \mathbb{P}} \left(1-\frac{1}{p_k} \right) \]

\textbf{Euljerjev izrek:}
\[\textrm{gcd}(a, m) = 1 \Leftrightarrow a^{\varphi(m)} \equiv_m 1; a \in \mathbb{Z}_m^*\]
\[a,m \in \mathbb{N} \wedge \textrm{gcd}(a, m) = 1 \Rightarrow a^{\varphi(m)} \equiv_m 1\]
\[a^{\varphi(m)} = 1 \text{ v } \mathbb{Z}_m^*\]

\textbf{Mali Fermatov izrek:} če je $m \in \mathbb{P}$ ($\varphi(m) = m-1$) in $\textrm{gcd}(a,m) = 1$, potem:
\[a^{m-1} \equiv_m 1\]

\section{RSA}
$A$ želi varno prejeti sporočilo od $B$.
\begin{itemize}
    \item $A$ izbere praštevili $p$ in $q$
    \item $A$ izračuna $n = pq$ in $\varphi = \varphi(n) = (p-1)(q-1)$
    \item $A$ izbere $e \in \mathbb{Z}_{\varphi}^*$, ki je tuje $\varphi$
    \item $A$ izračuna $d = e^{-1}$ (reši diofantsko enačbo $ex - \varphi y = 1$ za $x=d$ in $y$)
    \item $A$ javno objavi $(n, e)$ in si naskrivaj zapomni $d$
    \item $B$ sestavi sporočilo $m$
    \item $B$ izračuna $m' = m^e \textrm{ mod } n$
    \item $B$ pošlje $m'$
    \item $A$ izračuna $m'' = m'^d \textrm{ mod } n$ 
\end{itemize}
Izkaže se, da je $m''$ enak $m$

\section{Permutacije}
Permutacija množice $\Omega$ je bijektivna preslikava $\pi: \Omega \rightarrow \Omega$
$\textrm{Sym}(\Omega)$ je množica vseh permutacij na $\Omega$.
\[|\textrm{Sym}(\Omega)| = |\Omega|!\]
\[S_n = \textrm{Sym}(\Omega); \Omega = \{1,2,...,n\}\]

\subsubsection{Ciklična struktura}
Multimnožica doložin ciklov.

\begin{itemize}
    \item \textbf{negibne točke}: cilki dolžine 1
    \item \textbf{transpozicije}: cilki dolžine 2
    \item \textbf{k-cikli}: cilki dolžine k
\end{itemize}

\textbf{Ciklična premutacija} je taka premutacija kjer je največ en cikel dolžine več kot 1. (ostali pa so dolžine 1)

\subsubsection{Produkt permutacij}
\[(\pi \cdot \varphi)(\omega) = (\varphi \circ \pi)(\omega) = \varphi\left(\pi(\omega)\right) \]
\[\pi(\omega) = \omega^{\pi}\]
\[(\pi \cdot \varphi)(\omega) = \omega^{(\pi \cdot \varphi)} = \left(\omega^{\pi}\right)^{\varphi} = \omega^{\pi \varphi}\]


\textbf{Nosilec}

Naj bo $\pi \in \textrm{Sym}(\Omega)$.
\[\textrm{supp}(\pi) = \{\omega \in \Omega : \omega^{\pi} \neq \omega\}\]
\[\omega \in \textrm{supp}(\pi) \Leftrightarrow \omega^{\pi} \in \textrm{supp}(\pi) \]

$\pi, \varphi \in \textrm{Sym}(\Omega)$ sta \textbf{disjunktni} permutaciji, če
\[\textrm{supp}(\pi) \cap \textrm{supp}(\varphi) = \emptyset\]

Če sta permutaciji $\pi, \varphi \in \textrm{Sym}(\Omega)$ disjunktni, \textbf{komutirata} ($\pi \cdot \varphi = \varphi \cdot \pi$).

\subsubsection{Red permutacije}
Red permutacije $\alpha \in S_n$ je najmanjše število $k$, da je
\[\alpha^k = \textrm{id}\]

\subsubsection{Inverzije}
$\varphi \in S_n$, par števil $i,j$ ($1\leq i < j \leq n)$ je v \emph{inverzu} v permutaciji $\varphi$, če se v spodnji vrstici zapisa parmutacije $\varphi$ s tabelo pojavita v "napačnem" vrstnem redu: večje število je zapisano bolj levo od manjšega.

Število inverzi permutacije $\varphi$ označimo z $\textrm{inv}(\varphi)$.
\bigbreak
Denimo, da je $\varphi$ permutacija in $\tau$ transpozicija, potem velja:
\[\textrm{inv}(\varphi) \not\equiv_2 \textrm{inv}(\varphi \cdot \tau)\]
\bigbreak
Permutacija $\varphi \in S_n$ je soda $\Leftrightarrow$ $\textrm{inv}(\varphi)$ sodo.

Permutacija $\varphi \in S_n$ je liha $\Leftrightarrow$ $\textrm{inv}(\varphi)$ liho.
\bigbreak
Vsako permutacijo lahko zapišemo kot produkt transpozicij.
\[\pi = (13927)(4658) = (13)(19)(12)(17)(46)(45)(48)\]
\[(1\ 2\ 3\ ...\ n) = (1\ 2)(1\ 3)...(1\ n)\]

Permutacija je \textbf{soda}, če je $n$ lih.

Permutacija je \textbf{liha}, če je $n$ sod.
\end{multicols}
\end{document}
